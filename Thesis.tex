%---------------------------------------------------------------------------%
%-                                                                         -%
%-                           LaTeX Template                                -%
%-                                                                         -%
%---------------------------------------------------------------------------%
%- Copyright (C) Huangrui Mo <huangrui.mo@gmail.com> 
%- ------------------------------------声明---------------------------------%
%- 本模板基于中国科学院大学ucasthesis(mohuangrui版)进行修改,适用于苏州大学硕博
%- (学术/专业)学位论文要求,因时间和能力限制,暂时不支持同等学力版本,且本版本为
%- 非官方版本,各学院专业具体要求不同,可自行修改。
%- 具体使用可参考使用手册
%---------------------------------------------------------------------------%
%->> Document class declaration
%---------------------------------------------------------------------------%
\documentclass[twoside]{Style/sudathesis}%
%- Multiple optional arguments:
%- [oneside] 单面排版,适合作为节省页面的电子版。
%- [twoside] 双面排版,适合作为正式电子版。
%- [print] 预留了装订距离的双面排版,并会将超链接设为黑色,适合作为纸质打印版。
%- 打印前修改为print
%---------------------------------------------------------------------------%
%->> Document settings
%---------------------------------------------------------------------------%
\usepackage[super,list]{Style/artratex}% document settings
%- usage: \usepackage[option1,option2,...,optionN]{artratex}
%- Multiple optional arguments:
%- [<numbers|super|authoryear|alpha>]% 设置引用和参考文献的形式
%- 著者-出版年制(authoryear)为默认选项,不同文献样式和引用样式,如 
%- 著者-出版年制(authoryear)、顺序编码制(numbers)、上标顺序编码制(super)、
%- 字符编码制(alpha)可在 Thesis.tex 中对 artratex.sty 调用实现。
%- 默认使用格式满足学校要求。

\usepackage{Style/artracom}% user defined commands
%---------------------------------------------------------------------------%
%->> Document inclusion
%---------------------------------------------------------------------------%
%\includeonly{Tex/Chap_1,...,Tex/Chap_N}% selected files compilation
%---------------------------------------------------------------------------%
%->> Document content
%---------------------------------------------------------------------------%
%-
%-> Titlepage information
%-
%---------------------------------------------------------------------------%
%->> Titlepage information
%- 	!!!!!!!!!!!!!!!!!!!!!!!!注!!!意!!!!!!!!!!!!!!!!!!!!!!
%- 	!!!!!!!!!!!!!!!!!!!!!!!!注!!!意!!!!!!!!!!!!!!!!!!!!!!
%- !!!请注意,因模板调用要求不同,存在多次设置相同信息问题!!!
%- 	!!!!!!!!!!!!!!!!!!!!!!!!注!!!意!!!!!!!!!!!!!!!!!!!!!!
%- 	!!!!!!!!!!!!!!!!!!!!!!!!注!!!意!!!!!!!!!!!!!!!!!!!!!!
%---------------------------------------------------------------------------%
%-
%-> 中文封面信息
%-
%- 密级:只有涉密论文才填写,编者专业暂未要求;如果需要,可联系编者修改
\schoolid{10285}
\classid{42}
\schoolname[scale=1.0]{name}% 苏州大学
\type{(学术学位)}%学位类型
\schoollogo[scale=1.0]{logo}% 校徽
\title{苏州大学学位论文\LaTeX{}模板 {$~^{\pi}\pi^{\pi}$}}% 论文中文题目文中
\titlea{苏州大学\\学位论文\LaTeX{}模板 {$~^{\pi}\pi^{\pi}$}}% 论文中文题目封面页
\author{莫晃锐}% 论文作者
\advisor{刘青泉}% 指导教师:姓名 专业技术职务 工作单位(两位及以上指导教师使用\\换行)
%\advisor{刘青泉\\刘青泉~研究员~中国科学院理论力学研究所}% 指导教师:姓名 专业技术职务 工作单位
\degree{硕士}% 学位:学士、硕士、博士
\degreetype{理学}% 学位类别:理学、工学、工程、医学等
\major{物理学}% 专业名称
\direction{粒子物理}
\institute{物理与科学技术学院}% 院系名称
\date{2014~年~6~月}% 毕业日期:夏季为6月、冬季为12月
%-
%-> 英文封面信息
%-
\TITLE{\LaTeX{} Thesis Template\\ of \\ Soochow University {$~^{\pi}\pi^{\pi}$}}% 论文英文题目英文封面页
\TITLEA{\LaTeX{} Thesis Template of \\ Soochow University {$~^{\pi}\pi^{\pi}$}}% 论文英文题目中文封面页
\AUTHOR{Mo Huangrui}% 论文作者
\ADVISOR{Professor Liu Qingquan$^a$\\Associate Professor Liu Qingquan$^b$}% 指导教师
\DEGREE{Master}% 学位:Bachelor, Master, Doctor。封面据英文学位名称自动切换,需确保拼写准确
\DEGREETYPE{Natural Science}% 学位类别:Philosophy, Natural Science, Engineering, Economics, Agriculture 等
\MAJOR{Particle Physics and Nuclera Physics}% 专业名称
\INSTITUTE{$^a$School of Physics Science and Technology, Soochow University
	\\$^b$Institute of High Energy Physics, Chinese Academy of Sciences}% 院系名称
\DATE{June, 2014}% 毕业日期:夏季为June、冬季为December
%---------------------------------------------------------------------------%

%-> 盲审封面信息
\titlebr{苏州大学学位论文\LaTeX{}模板 {$~^{\pi}\pi^{\pi}$}\\}% 盲审中文题目
\brstudentnum{2022420000}%申请人学号
\brmajor{物理学}%专业名称
\brdirection{粒子物理学}%研究方向
\brkeywords{苏州大学,学位论文,\LaTeX{}模板\\}%论文关键词

% 设置基本信息
\begin{document}
%-
%-> Frontmatter: title page, abstract, content list, symbol list, preface
%-> Frontmatter:扉页、摘要、内容列表、符号列表、前言
%-
\frontmatter% 调用摘要
%---------------------------------------------------------------------------%
%->> Frontmatter
%---------------------------------------------------------------------------%
%-
%-> 生成封面
%-
%\blindreview%生成盲审封面
\maketitle% 生成中文封面
\MAKETITLE% 生成英文封面
%-
%-> 独创声明
%-
\makedeclaration% 生成声明页
%-
%-> 中文摘要
%-
\abstract{本文是苏州大学学位论文模板sudathesis的使用说明文档。主要内容为介绍\LaTeX{}文档类sudathesis的用法,以及如何使用\LaTeX{}快速高效地撰写学位论文。\textbf{编译前请前往\href{https://github.com/sudathesis/sudathesis/wiki}{\texttt{sudathesis Wiki}} 查看基础介绍}}

\keywords{苏州大学,学位论文,\LaTeX{}模板}% 中文关键词

\absauthor{莫晃锐}
\absadvisor{刘青泉\\莫\quad 锐}%同封面页设置
%-
%-> 注释上方两行,打开下方两行用于盲审
%-
%\absauthor{***}
%\absadvisor{***}%同封面页设置

\makeabstract

%-
%-> 英文摘要
%-
\ABSTRACT{This paper is a help documentation for the \LaTeX{} class sudathesis, which is  a thesis template for Soochow University. The main content is about how to use the sudathesis, as well as how to write thesis efficiently by using \LaTeX{}.}

\KEYWORDS{Soochow University(SUDA), Thesis, \LaTeX{} Template}% 英文关键词

\ABSAUTHOR{Mo Huangrui}
\ABSADVISOR{Liu Qingquan\\Mo rui\\AAAA}%同封面页设置
%-
%-> 注释上方两行,打开下方两行用于盲审
%-
%\ABSAUTHOR{***}
%\ABSADVISOR{***}%同封面页设置

\MAKEABSTRACT


%---------------------------------------------------------------------------%
% 摘要
{% content list region
\linespread{1.2}% 行间距
\intobmk*{\cleardoublepage}{\contentsname}% 添加书签链接
\tableofcontents% 正文目录,不包含中英文摘要
\intobmk*{\cleardoublepage}{\listfigurename}% 添加书签链接
\listoffigures% 插图索引
\intobmk*{\cleardoublepage}{\listtablename}% 添加书签链接
\listoftables% 表格索引
}
\input{Tex/Prematter}% 符号表、前言内容
%-
%-> Mainmatter
%-> 正文主体
%-
\mainmatter% 调用正文格式
\chapter{引言}\label{chap:introduction}

\section{研究背景}

多数的苏州大学毕业生在毕业论文排版时选择临时寻找\LaTeX{}模板进行修改,需要耗费不必要的研究修改时间。同时,各类模板修改难易程度不同,效果差异大。在发现中国科学院大学ucasthesis(莫晃锐版)\LaTeX{}模板的优异性后,对其修改适配于苏州大学学位论文编写格式要求,其中大部分底层代码和逻辑继承前者。

当前sudathesis模板满足最新的苏州大学学位论文撰写要求和封面设定。兼顾操作系统:Windows,Linux,MacOS和\LaTeX{}编译引擎:pdflatex,xelatex,lualatex。支持中文书签、中文渲染、中文粗体显示、拷贝PDF中的文本到其他文本编辑器等特性。此外,对模板的文档结构进行了精心设计,撰写了编译脚本提高模板的易用性和使用效率。

sudathesis的目标节省多数人修改\LaTeX{}模板时间,提高学位论文的撰写速度。 同时,sudathesis继承了ucasthesis模板整洁一致的代码结构和扼要的注解。

因为\LaTeX{}使用经验和涉猎层次不一,sudathesis基于\LaTeX{}格式与内容分离的特色,封装底层排版代码,设置简单的接口,降低使用难度,提升使用舒适度。对于初学者而言,使用此模板撰写学位论文本质上不存在技术性的困难。同时,针对\LaTeX{}撰写论文的一些主要难题,如制图、制表、文献索引等,将结合使用情况进行了详细说明,并提供了相应的代码替换方案,具体将在后续章节中进行阐述。


\section{系统要求}\label{sec:system}

\href{https://github.com/sudathesis/sudathesis}{\texttt{sudathesis}} 宏包可以在目前主流的 \href{https://en.wikibooks.org/wiki/LaTeX/Introduction}{\LaTeX{}} 编译系统中使用,如\TeX{}Live和MiK\TeX{}。因C\TeX{}套装已停止维护,\textbf{不再建议使用} (请勿混淆C\TeX{}套装与ctex宏包。C\TeX{}套装是集成了许多\LaTeX{}组件的\LaTeX{}编译系统。 \href{https://ctan.org/pkg/ctex?lang=en}{ctex} 宏包如同sudathesis,是\LaTeX{}命令集,其维护状态活跃,并被主流的\LaTeX{}编译系统默认集成,是几乎所有\LaTeX{}中文文档的核心架构)。推荐的 \href{https://en.wikibooks.org/wiki/LaTeX/Installation}{\LaTeX{}编译系统} 和 \href{https://en.wikibooks.org/wiki/LaTeX/Installation}{\LaTeX{}文本编辑器} 为
\begin{center}
    \begin{tabular}{lcc}
        \hline
        操作系统 & \LaTeX{}编译系统 & \LaTeX{}文本编辑器\\
        \hline
        Linux & \href{https://www.tug.org/texlive/acquire-netinstall.html}{\TeX{}Live Full} & \href{http://texstudio.sourceforge.net/}{TeXstudio} 或 Vim\\
        MacOS & \href{https://www.tug.org/mactex/}{Mac\TeX{} Full} & \href{http://texstudio.sourceforge.net/}{TeXstudio} 或 \href{https://www.texpad.com/}{Texpad for Mac}\\
        Windows & \href{https://www.tug.org/texlive/acquire-netinstall.html}{\TeX{}Live Full} 或 \href{https://miktex.org/download}{MiK\TeX{}} & \href{http://texstudio.sourceforge.net/}{TeXstudio}\\
        \hline
    \end{tabular}
\end{center}

\LaTeX{}编译系统,如\TeX{}Live(Mac\TeX{}为针对MacOS的\TeX{}Live),用于提供编译环境,\LaTeX{}文本编辑器 (如Texmaker) 用于编辑\TeX{}源文件。请从各软件官网下载安装程序,勿使用不明程序源。\textbf{\LaTeX{}编译系统和\LaTeX{}编辑器分别安装成功后,即完成了\LaTeX{}的系统配置},无需其他手动干预和配置。若系统原带有旧版的\LaTeX{}编译系统并想安装新版,请\textbf{先卸载干净旧版再安装新版}。

\section{问题反馈}

sudathesis基于ucasthesis对苏州大学毕业论文要求进行排版适配,未对底层逻辑和代码进行大篇幅修改,因此大多数常见问题可以前往ucasthesis的 \href{https://github.com/mohuangrui/ucasthesis/wiki/%E5%B8%B8%E8%A7%81%E9%97%AE%E9%A2%98}{问题反馈} 查看。
对于修改的sudathesis模板中出现的问题可以反馈至\href{https://github.com/sudathesis/sudathesis/issues}{sudathesis Issues},在力所能及的情况下将及时修改完善。

\section{模板下载}

\begin{center}
    \href{https://github.com/sudathesis/sudathesis}{Github/sudathesis}: \url{https://github.com/sudathesis/sudathesis}
\end{center}
 %引言
\chapter{模板文件介绍}\label{chap:texintro}

基于文件分类管理原则便于修改与操作的属性,本模板对sudathesis的框架和文件体系进行分类系统管理。对于初次使用者,\LaTeX{}编译后的众多的文件目录不知所云,在阅读完模板文件介绍后,将会轻松使用。除此之外,\LaTeX{}的使用方法可以通过阅读相关资料如\LaTeX{} Wikibook \citep{wikibook2014latex} 增添使用技巧。


\section{安装与操作}

\begin{enumerate}
    \item 安装软件:根据所用操作系统和章节~\ref{sec:system}中的信息安装\LaTeX{}编译环境。
    \item 获取模板:下载 \href{https://github.com/sudathesis/sudathesis}{\texttt{sudathesis}} 模板并解压。sudathesis模板不仅提供了相应的类文件,同时也提供了包括参考文献等在内的完成学位论文的一切要素,所以,下载时,推荐下载整个sudathesis文件夹,而不是单独的文档类。
    \item 编译模板:
        \begin{enumerate}
            \item Windows:双击运行artratex.bat脚本。
            \item Linux或MacOS: { \verb|terminal| -> \verb|chmod +x ./artratex.sh| -> \verb|./artratex.sh xa|}
            \item \color{red}任意系统:都可使用\LaTeX{}编辑器打开Thesis.tex文件并选择xelatex编译引擎进行编译。\\
            具体编译流程为 Xelatex -> Bibtex -> Xelatex -> Xelatex,即后文所提方法。
        \end{enumerate}
    \item 错误处理:若编译中遇到了问题,请先查看“常见问题”(章节~\ref{sec:qa})。
\end{enumerate}

编译完成即可获得本PDF说明文档。而这也完成了学习使用sudathesis撰写论文的一半进程。

\section{文档目录简介}

\subsection{Thesis.tex}

Thesis.tex为主文档,其设计和规划了论文的整体框架,通过对其的阅读可以了解整个论文框架的搭建。

\subsection{编译脚本}

\begin{itemize}
    \item Windows:双击Dos脚本artratex.bat可得全编译后的PDF文档,其存在是为了帮助不了解\LaTeX{}编译过程的初学者跨过编译这第一道坎,请勿通过邮件传播和接收此脚本,以防范Dos脚本的潜在风险。
    \item Linux或MacOS:在terminal中运行
        \begin{itemize}
            \item \verb|./artratex.sh xa|:获得全编译后的PDF文档
            \item \verb|./artratex.sh x|:快速编译,不会生成文献引用
        \end{itemize}
\end{itemize}

全编译指运行 \verb|Xelatex+Bibtex+Xelatex+Xelatex| 以正确生成所有的引用链接,如目录,参考文献及引用等。在写作过程中若无添加新的引用,则可用快速编译,即只运行一遍\LaTeX{}编译引擎以减少编译时间。

\subsection{Tmp文件夹}

运行编译脚本后,编译所生成的文档皆存于Tmp文件夹内,包括编译得到的PDF文档,其存在是为了保持工作空间的整洁,因为好的心情是很重要的。

\subsection{Style文件夹}

包含ucasthesis文档类的定义文件和配置文件,通过对它们的修改可以实现特定的模版设定。

\begin{enumerate}
    \item sudathesis.cls:文档类定义文件,论文的最核心的格式即通过它来定义的。
    \item sudathesis.cfg:文档类配置文件,设定如目录显示为“目~录”而非“目录”。
    \item artratex.sty: 常用宏包及文档设定,如参考文献样式、文献引用样式、页眉页脚设定等。这些功能具有开关选项,常只需在Thesis.tex中进行启用即可,一般无需修改artratex.sty本身。
    \item artracom.sty:自定义命令以及添加宏包的推荐放置位置。
\end{enumerate}

\subsection{Tex文件夹}

文件夹内为论文的所有实体内容,正常情况下,这也是\textbf{使用sudathesis撰写学位论文时,主要关注和修改的一个位置,注:所有文件都必须采用UTF-8编码,否则编译后将出现乱码文本},详细分类介绍如下:

\begin{itemize}
    \item Frontinfo.tex:为论文中英文封面信息。\textbf{论文封面会根据英文学位名称如Bachelor,Master,Doctor, Postdoctor 自动切换为相应的格式}。
    \item Frontmatter.tex:为论文前言内容如中英文摘要等。
    \item Mainmatter.tex:索引需要出现的Chapter。开始写论文时,可以只索引当前章节,以快速编译查看,当论文完成后,再对所有章节进行索引即可。
    \item Chap{\_}xxx.tex:为论文主体的各章,可根据需要添加和撰写。\textbf{添加新章时,可拷贝一个已有的章文件再重命名,以继承文档的 UTF8 编码}。
    \item Appendix.tex:为附录内容。
    \item Backmatter.tex:为发表文章信息和致谢部分等。
\end{itemize}

\subsection{Img文件夹}

用于放置论文中所需要的图类文件,支持格式有:.eps, .jpg, .png, .pdf。其中,\verb|logo.png|苏州大学校徽、\verb|name.png|苏州大学校名图片格式。若图片众多,也可为各章节图片建立子目录;但若命名规则合理,图片查询亦是十分方便。

\subsection{Biblio文件夹}

\begin{enumerate}
    \item ref.bib:参考文献信息库。
\end{enumerate}


 %第一章
\chapter{模板使用说明}\label{chap:guide}

\section{字体、图表、数学公式、参考文献等功能}

\subsection{文档内字体切换方法}
具体使用代码可查看tex文档下方示例,也可以前往\href{https://github.com/sudathesis/sudathesis/wiki/%E5%AD%97%E4%BD%93%E5%AD%97%E5%8F%B7#%E5%B8%B8%E7%94%A8%E5%AD%97%E4%BD%93%E4%BF%AE%E6%94%B9}{常用字体修改}查看。
\begin{itemize}
	\item 宋体:论文模板 或 \textrm{论文模板}
	\item 粗宋体:{\bfseries 论文模板} 或 \textbf{论文模板}
	\item 黑体:{\sffamily 论文模板} 或 \textsf{论文模板}
	\item 粗黑体:{\bfseries\sffamily 论文模板} 或 \textsf{\bfseries 论文模板}
	\item 仿宋:{\ttfamily 论文模板} 或 \texttt{论文模板}
	\item 粗仿宋:{\bfseries\ttfamily 论文模板} 或 \texttt{\bfseries 论文模板}
	\item 楷体:{\itshape 论文模板} 或 \textit{论文模板}
	\item 粗楷体:{\bfseries\itshape 论文模板} 或 \textit{\bfseries 论文模板}
\end{itemize}

\subsection{表格}

请见表~\ref{tab:sample}。
\begin{table}[!htbp]
	\bicaption{这是一个样表。}{This is a sample table.}
	\label{tab:sample}
	\centering
	\footnotesize% fontsize
	\setlength{\tabcolsep}{4pt}% column separation
	\renewcommand{\arraystretch}{1.2}%row space 
	\begin{tabular}{lcccccccc}
		\hline
		行号 & \multicolumn{8}{c}{跨多列的标题}\\
		%\cline{2-9}% partial hline from column i to column j
		\hline
		Row 1 & $1$ & $2$ & $3$ & $4$ & $5$ & $6$ & $7$ & $8$\\
		Row 2 & $1$ & $2$ & $3$ & $4$ & $5$ & $6$ & $7$ & $8$\\
		Row 3 & $1$ & $2$ & $3$ & $4$ & $5$ & $6$ & $7$ & $8$\\
		Row 4 & $1$ & $2$ & $3$ & $4$ & $5$ & $6$ & $7$ & $8$\\
		\hline
	\end{tabular}
\end{table}

基础制表案例及制表方法请见 \href{https://github.com/sudathesis/sudathesis/wiki#%E8%A1%A8%E6%A0%BC%E8%AE%BE%E7%BD%AE}{sudathesis表格设置}表格常用形式。制图制表的更多范例,请见 \href{https://en.wikibooks.org/wiki/LaTeX/Tables}{WiKibook Tables}。

\subsection{图片插入}

论文中图片的插入通常分为单图和多图,下面分别加以介绍:

单图插入:假设插入名为\verb|c06h06|(后缀可以为.jpg、.png、.pdf,下同)的图片,其效果如图~\ref{fig:c06h06}。
\begin{figure}[!htbp]
	\centering
	\includegraphics[width=0.40\textwidth]{c06h06}
	\caption{Q判据等值面图,同时测试一下一个很长的标题,比如这真的是一个很长很长很长很长很长很长很长很长的标题。}
	\label{fig:c06h06}
\end{figure}

如果插图的空白区域过大,以图片\verb|c06h06|为例,自动裁剪如图~\ref{fig:c06h06_trim}。
\begin{figure}[!htbp]
	\centering
	%trim option's parameter order: left bottom right top
	\includegraphics[trim = 60mm 80mm 60mm 60mm, clip, width=0.40\textwidth]{c06h06}
	\caption{激波圆柱作用。}
	\label{fig:c06h06_trim}
\end{figure}

多图的插入如图~\ref{fig:oaspl},多图不应在子图中给文本子标题,只要给序号,并在主标题中进行引用说明。
\begin{figure}[!htbp]
	\centering
	\begin{subfigure}[b]{0.35\textwidth}
		\includegraphics[width=\textwidth]{oaspl_a}
		\caption{}
		\label{fig:oaspl_a}
	\end{subfigure}%
	~% add desired spacing
	\begin{subfigure}[b]{0.35\textwidth}
		\includegraphics[width=\textwidth]{oaspl_b}
		\caption{}
		\label{fig:oaspl_b}
	\end{subfigure}
	\\% line break
	\begin{subfigure}[b]{0.35\textwidth}
		\includegraphics[width=\textwidth]{oaspl_c}
		\caption{}
		\label{fig:oaspl_c}
	\end{subfigure}%
	~% add desired spacing
	\begin{subfigure}[b]{0.35\textwidth}
		\includegraphics[width=\textwidth]{oaspl_d}
		\caption{}
		\label{fig:oaspl_d}
	\end{subfigure}
	\caption{总声压级。(a) 这是子图说明信息,(b) 这是子图说明信息,(c) 这是子图说明信息,(d) 这是子图说明信息。}
	\label{fig:oaspl}
\end{figure}


\subsection{数学公式}

比如Navier-Stokes方程(方程~\eqref{eq:ns}):
\begin{equation} \label{eq:ns}
    \adddotsbeforeeqnnum%
    \begin{cases}
        \frac{\partial \rho}{\partial t} + \nabla\cdot(\rho\Vector{V}) = 0 \ \mathrm{times\ math\ test: 1,2,3,4,5}, 1,2,3,4,5\\
        \frac{\partial (\rho\Vector{V})}{\partial t} + \nabla\cdot(\rho\Vector{V}\Vector{V}) = \nabla\cdot\Tensor{\sigma} \ \text{times text test: 1,2,3,4,5}\\
        \frac{\partial (\rho E)}{\partial t} + \nabla\cdot(\rho E\Vector{V}) = \nabla\cdot(k\nabla T) + \nabla\cdot(\Tensor{\sigma}\cdot\Vector{V})
    \end{cases}
\end{equation}
\begin{equation}
    \adddotsbeforeeqnnum%
    \frac{\partial }{\partial t}\int\limits_{\Omega} u \, \mathrm{d}\Omega + \int\limits_{S} \unitVector{n}\cdot(u\Vector{V}) \, \mathrm{d}S = \dot{\phi}
\end{equation}
\[
    \begin{split}
        \mathcal{L} \{f\}(s) &= \int _{0^{-}}^{\infty} f(t) e^{-st} \, \mathrm{d}t, \ 
        \mathscr{L} \{f\}(s) = \int _{0^{-}}^{\infty} f(t) e^{-st} \, \mathrm{d}t\\
        \mathcal{F} {\bigl (} f(x+x_{0}) {\bigr )} &= \mathcal{F} {\bigl (} f(x) {\bigr )} e^{2\pi i\xi x_{0}}, \ 
        \mathscr{F} {\bigl (} f(x+x_{0}) {\bigr )} = \mathscr{F} {\bigl (} f(x) {\bigr )} e^{2\pi i\xi x_{0}}
    \end{split}
\]

数学公式常用命令请见 \href{https://en.wikibooks.org/wiki/LaTeX/Mathematics}{WiKibook Mathematics}。artracom.sty中对一些常用数据类型如矢量矩阵等进行了封装,这样的好处是如有一天需要修改矢量的显示形式,只需单独修改artracom.sty中的矢量定义即可实现全文档的修改。

\subsection{数学环境}

\begin{axiom}
   这是一个公理。 
\end{axiom}
\begin{theorem}
   这是一个定理。 
\end{theorem}
\begin{lemma}
   这是一个引理。 
\end{lemma}
\begin{corollary}
   这是一个推论。 
\end{corollary}
\begin{assertion}
   这是一个断言。 
\end{assertion}
\begin{proposition}
   这是一个命题。 
\end{proposition}
\begin{proof}
    这是一个证明。
\end{proof}
\begin{definition}
    这是一个定义。
\end{definition}
\begin{example}
    这是一个例子。
\end{example}
\begin{remark}
    这是一个注。
\end{remark}


\subsection{算法}

如见算法~\ref{alg:euclid},详细使用方法请参见文档 \href{https://ctan.org/pkg/algorithmicx?lang=en}{algorithmicx}。

\begin{algorithm}[!htbp]
    \small
    \caption{Euclid's algorithm}\label{alg:euclid}
    \begin{algorithmic}[1]
        \Procedure{Euclid}{$a,b$}\Comment{The g.c.d. of a and b}
        \State $r\gets a\bmod b$
        \While{$r\not=0$}\Comment{We have the answer if r is 0}
        \State $a\gets b$
        \State $b\gets r$
        \State $r\gets a\bmod b$
        \EndWhile\label{euclidendwhile}
        \State \textbf{return} $b$\Comment{The gcd is b}
        \EndProcedure
    \end{algorithmic}
\end{algorithm}

\subsection{参考文献引用}

参考文献引用过程以实例进行介绍,假设需要引用名为"Document Preparation System"的文献,步骤如下:

1)使用Google Scholar搜索Document Preparation System,在目标条目下点击Cite,展开后选择Import into BibTeX打开此文章的BibTeX索引信息,将它们copy添加到ref.bib文件中(此文件位于Biblio文件夹下)。

2)索引第一行 \verb|@article{lamport1986document,|中 \verb|lamport1986document| 即为此文献的label (\textbf{中文文献也必须使用英文label},一般遵照:姓氏拼音+年份+标题第一字拼音的格式),想要在论文中索引此文献,有两种索引类型:

文本类型:\verb|\citet{lamport1986document}|。正如此处所示 \citet{lamport1986document}; 

括号类型:\verb|\citep{lamport1986document}|。正如此处所示 \citep{lamport1986document}。

\textbf{多文献索引用英文逗号隔开}:

\verb|\citep{lamport1986document, chu2004tushu, chen2005zhulu}|。正如此处所示 \citep{lamport1986document, chu2004tushu, chen2005zhulu}

更多例子如:

\citet{walls2013drought} 根据 \citet{betts2005aging} 的研究,首次提出...。其中关于... \citep{walls2013drought, betts2005aging},是当前中国...得到迅速发展的研究领域 \citep{chen1980zhongguo, bravo1990comparative}。引用同一著者在同一年份出版的多篇文献时,在出版年份之后用
英文小写字母区别,如:\citep{yuan2012lana, yuan2012lanb, yuan2012lanc} 和 \citet{yuan2012lana, yuan2012lanb, yuan2012lanc}。同一处引用多篇文献时,按出版年份由近及远依次标注。例如 \citep{chen1980zhongguo, stamerjohanns2009mathml, hls2012jinji, niu2013zonghe}。

使用著者-出版年制(authoryear)式参考文献样式时,中文文献必须在BibTeX索引信息的 \textbf{key} 域(请参考ref.bib文件)填写作者姓名的拼音,才能使得文献列表按照拼音排序。参考文献表中的条目(不排序号),先按语种分类排列,语种顺 序是:中文、日文、英文、俄文、其他文种。然后,中文按汉语拼音字母顺序排列,日文按第一著者的姓氏笔画排序,西文和 俄文按第一著者姓氏首字母顺序排列。如中 \citep{niu2013zonghe}、日 \citep{Bohan1928}、英 \citep{stamerjohanns2009mathml}、俄 \citep{Dubrovin1906}。

如此,即完成了文献的索引,请查看下本文档的参考文献一章,看看是不是就是这么简单呢?是的,就是这么简单!

不同文献样式和引用样式,如著者-出版年制(authoryear)、顺序编码制(numbers)、上标顺序编码制(super)可在Thesis.tex中对artratex.sty调用实现,详见 \href{https://github.com/mohuangrui/ucasthesis/wiki}{ucasthesis 知识小站之文献样式}

%若在上标顺序编码制(super)模式下,希望在特定位置将上标改为嵌入式标,可使用 \citetns{niu2013zonghe,stamerjohanns2009mathml} 和 \citepns{niu2013zonghe,stamerjohanns2009mathml}。

参考文献索引的更多知识,请见 \href{https://en.wikibooks.org/wiki/LaTeX/Bibliography_Management}{WiKibook Bibliography}。\nocite{*}% 使文献列表显示所有参考文献(包括未引用文献)

\section{常见使用问题}\label{sec:qa}

\begin{enumerate}
    \item 模板在发布前,已在Windows,Linux,MacOS系统上测试通过。下载模板后,若编译出现错误,则请见 \href{https://github.com/mohuangrui/ucasthesis/wiki}{ucasthesis知识小站} 的 \href{https://github.com/mohuangrui/ucasthesis/wiki/%E7%BC%96%E8%AF%91%E6%8C%87%E5%8D%97}{编译指南}。

    \item 模板文档的编码为UTF-8编码。所有文件都必须采用UTF-8编码,否则编译后生成的文档将出现乱码文本。若出现文本编辑器无法打开文档或打开文档乱码的问题,请检查编辑器对UTF-8编码的支持。如果使用WinEdt作为文本编辑器(\textbf{不推荐使用}),应在其Options -> Preferences -> wrapping选项卡下将两种Wrapping Modes中的内容:
        
        TeX;HTML;ANSI;ASCII|DTX...
        
        修改为:TeX;\textbf{UTF-8|ACP;}HTML;ANSI;ASCII|DTX...
        
        同时,取消Options -> Preferences -> Unicode中的Enable ANSI Format。

    \item 推荐选择xelatex或lualatex编译引擎编译中文文档。编译脚本的默认设定为xelatex编译引擎。你也可以选择不使用脚本编译,如直接使用 \LaTeX{}文本编辑器编译。注:\LaTeX{}文本编辑器编译的默认设定为pdflatex编译引擎,若选择xelatex或lualatex编译引擎,请进入下拉菜单选择。为正确生成引用链接和参考文献,需要进行\textbf{全编译}。

    \item Texmaker使用简介
        \begin{enumerate}
            \footnotesize
            \item 使用 Texmaker “打开 (Open)” Thesis.tex。
            \item 菜单 “选项 (Options)” -> “设置当前文档为主文档 (Define as Master Document)”
            \item 菜单 “自定义 (User)” -> “自定义命令 (User Commands)” -> “编辑自定义命令 (Edit User Commands)” -> 左侧选择 “command 1”,右侧 “菜单项 (Menu Item)” 填入 Auto Build -> 点击下方“向导 (Wizard)” -> “添加 (Add)”: xelatex + bibtex + xelatex + xelatex + pdf viewer -> 点击“完成 (OK)”
            \item 使用 Auto Build 编译带有未生成引用链接的源文件,可以仅使用 xelatex 编译带有已经正确生成引用链接的源文件。
            \item 编译完成,“查看(View)” PDF,在PDF中 “ctrl+click” 可链接到相对应的源文件。
        \end{enumerate}
    
    \item 模版的设计可能地考虑了适应性。致谢等所有条目都是通过最为通用的

        \verb+\chapter{item name}+  and \verb+\section*{item name}+

        来显式实现的 (请观察Backmatter.tex),从而可以随意添加,放置,和修改,如同一般章节。对于图表目录名称则可在ucasthesis.cfg中进行修改。

    \item 设置文档样式: 在artratex.sty中搜索关键字定位相应命令,然后修改
        \begin{enumerate}
            \item 正文行距:启用和设置 \verb|\linespread{1.5}|,默认1.5倍行距。
            \item 参考文献行距:修改 \verb|\setlength{\bibsep}{0.0ex}|
            \item 目录显示级数:修改 \verb|\setcounter{tocdepth}{2}|
            \item 文档超链接的颜色及其显示:修改 \verb|\hypersetup|
        \end{enumerate}


\end{enumerate}


 %第二章
%- 请继续添加
%-
%-> Appendix
%-
\cleardoublepage%
\appendix% 调用附录格式
\input{Tex/Appendix}% 附录内容
%-
%-> Backmatter: bibliography, glossary, index
%-> Backmatter: 参考书目、词汇表、索引
%-
\backmatter% initialize the environment
\intotoc*{\cleardoublepage}{\bibname}% 添加书签链接
\artxifstreq{\artxbib}{bibtex}{% 支持 bibtex
    \bibliography{Biblio/ref}% bibliography
}{%
    \printbibliography% bibliography
}

%---------------------------------------------------------------------------%
%->> Backmatter
%---------------------------------------------------------------------------%
\chapter[致谢]{致\quad 谢}\chaptermark{致\quad 谢}% syntax: \chapter[目录]{标题}\chaptermark{页眉}
%\thispagestyle{noheaderstyle}% 如果需要移除当前页的页眉
%\pagestyle{noheaderstyle}% 如果需要移除整章的页眉

非常感谢ucasthesis的作者mohuangrui,他的ucasthesis模板代码逻辑清晰,结构简单,为我调整适配苏州大学学位论文编写要求的\LaTeX{}模板sudathesis相关工作节省很多时间。同时,对于mohuangrui所提及感谢的所有人也表示感谢。基于他们的智慧、辛勤付出和非凡工作,\LaTeX{}知识薄弱的我勉强完成对苏州大学学位论文\LaTeX{}模板sudathesis的修改工作。此外在自己的使用\LaTeX{}过程中,离不开众多在开源社区提供优秀教程和有效指导的贡献者,在此一致表示感谢!





\chapter{作者简历及攻读学位期间发表的学术论文与研究成果}

\textbf{本科生无需此部分}。

\section*{作者简历:}


\subsection*{ucasthesis作者}

莫晃锐,湖南省湘潭县人,中国科学院力学研究所硕士研究生。

\section*{已发表(或正式接受)的学术论文:}

{
\setlist[enumerate]{}% restore default behavior
\begin{enumerate}[nosep]
    \item ucasthesis: A LaTeX Thesis Template for the University of Chinese Academy of Sciences, 2014.
\end{enumerate}
}

\section*{申请或已获得的专利:}

(无专利时此项不必列出)

\section*{参加的研究项目及获奖情况:}

可以随意添加新的条目或是结构。

\cleardoublepage[plain]% 让文档总是结束于偶数页,可根据需要设定页眉页脚样式,如 [noheaderstyle]
%---------------------------------------------------------------------------%
% 致谢和成果总结
\end{document}
%---------------------------------------------------------------------------%

