\chapter{模板使用说明}\label{chap:guide}

\section{字体、图表、数学公式、参考文献等功能}

\subsection{文档内字体切换方法}
具体使用代码可查看tex文档下方示例,也可以前往\href{https://github.com/sudathesis/sudathesis/wiki/%E5%AD%97%E4%BD%93%E5%AD%97%E5%8F%B7#%E5%B8%B8%E7%94%A8%E5%AD%97%E4%BD%93%E4%BF%AE%E6%94%B9}{常用字体修改}查看。
\begin{itemize}
	\item 宋体:论文模板 或 \textrm{论文模板}
	\item 粗宋体:{\bfseries 论文模板} 或 \textbf{论文模板}
	\item 黑体:{\sffamily 论文模板} 或 \textsf{论文模板}
	\item 粗黑体:{\bfseries\sffamily 论文模板} 或 \textsf{\bfseries 论文模板}
	\item 仿宋:{\ttfamily 论文模板} 或 \texttt{论文模板}
	\item 粗仿宋:{\bfseries\ttfamily 论文模板} 或 \texttt{\bfseries 论文模板}
	\item 楷体:{\itshape 论文模板} 或 \textit{论文模板}
	\item 粗楷体:{\bfseries\itshape 论文模板} 或 \textit{\bfseries 论文模板}
\end{itemize}

\subsection{表格}

请见表~\ref{tab:sample}。
\begin{table}[!htbp]
	\bicaption{这是一个样表。}{This is a sample table.}
	\label{tab:sample}
	\centering
	\footnotesize% fontsize
	\setlength{\tabcolsep}{4pt}% column separation
	\renewcommand{\arraystretch}{1.2}%row space 
	\begin{tabular}{lcccccccc}
		\hline
		行号 & \multicolumn{8}{c}{跨多列的标题}\\
		%\cline{2-9}% partial hline from column i to column j
		\hline
		Row 1 & $1$ & $2$ & $3$ & $4$ & $5$ & $6$ & $7$ & $8$\\
		Row 2 & $1$ & $2$ & $3$ & $4$ & $5$ & $6$ & $7$ & $8$\\
		Row 3 & $1$ & $2$ & $3$ & $4$ & $5$ & $6$ & $7$ & $8$\\
		Row 4 & $1$ & $2$ & $3$ & $4$ & $5$ & $6$ & $7$ & $8$\\
		\hline
	\end{tabular}
\end{table}

基础制表案例及制表方法请见 \href{https://github.com/sudathesis/sudathesis/wiki#%E8%A1%A8%E6%A0%BC%E8%AE%BE%E7%BD%AE}{sudathesis表格设置}表格常用形式。制图制表的更多范例,请见 \href{https://en.wikibooks.org/wiki/LaTeX/Tables}{WiKibook Tables}。

\subsection{图片插入}

论文中图片的插入通常分为单图和多图,下面分别加以介绍:

单图插入:假设插入名为\verb|c06h06|(后缀可以为.jpg、.png、.pdf,下同)的图片,其效果如图~\ref{fig:c06h06}。
\begin{figure}[!htbp]
	\centering
	\includegraphics[width=0.40\textwidth]{c06h06}
	\caption{Q判据等值面图,同时测试一下一个很长的标题,比如这真的是一个很长很长很长很长很长很长很长很长的标题。}
	\label{fig:c06h06}
\end{figure}

如果插图的空白区域过大,以图片\verb|c06h06|为例,自动裁剪如图~\ref{fig:c06h06_trim}。
\begin{figure}[!htbp]
	\centering
	%trim option's parameter order: left bottom right top
	\includegraphics[trim = 60mm 80mm 60mm 60mm, clip, width=0.40\textwidth]{c06h06}
	\caption{激波圆柱作用。}
	\label{fig:c06h06_trim}
\end{figure}

多图的插入如图~\ref{fig:oaspl},多图不应在子图中给文本子标题,只要给序号,并在主标题中进行引用说明。
\begin{figure}[!htbp]
	\centering
	\begin{subfigure}[b]{0.35\textwidth}
		\includegraphics[width=\textwidth]{oaspl_a}
		\caption{}
		\label{fig:oaspl_a}
	\end{subfigure}%
	~% add desired spacing
	\begin{subfigure}[b]{0.35\textwidth}
		\includegraphics[width=\textwidth]{oaspl_b}
		\caption{}
		\label{fig:oaspl_b}
	\end{subfigure}
	\\% line break
	\begin{subfigure}[b]{0.35\textwidth}
		\includegraphics[width=\textwidth]{oaspl_c}
		\caption{}
		\label{fig:oaspl_c}
	\end{subfigure}%
	~% add desired spacing
	\begin{subfigure}[b]{0.35\textwidth}
		\includegraphics[width=\textwidth]{oaspl_d}
		\caption{}
		\label{fig:oaspl_d}
	\end{subfigure}
	\caption{总声压级。(a) 这是子图说明信息,(b) 这是子图说明信息,(c) 这是子图说明信息,(d) 这是子图说明信息。}
	\label{fig:oaspl}
\end{figure}


\subsection{数学公式}

比如Navier-Stokes方程(方程~\eqref{eq:ns}):
\begin{equation} \label{eq:ns}
    \adddotsbeforeeqnnum%
    \begin{cases}
        \frac{\partial \rho}{\partial t} + \nabla\cdot(\rho\Vector{V}) = 0 \ \mathrm{times\ math\ test: 1,2,3,4,5}, 1,2,3,4,5\\
        \frac{\partial (\rho\Vector{V})}{\partial t} + \nabla\cdot(\rho\Vector{V}\Vector{V}) = \nabla\cdot\Tensor{\sigma} \ \text{times text test: 1,2,3,4,5}\\
        \frac{\partial (\rho E)}{\partial t} + \nabla\cdot(\rho E\Vector{V}) = \nabla\cdot(k\nabla T) + \nabla\cdot(\Tensor{\sigma}\cdot\Vector{V})
    \end{cases}
\end{equation}
\begin{equation}
    \adddotsbeforeeqnnum%
    \frac{\partial }{\partial t}\int\limits_{\Omega} u \, \mathrm{d}\Omega + \int\limits_{S} \unitVector{n}\cdot(u\Vector{V}) \, \mathrm{d}S = \dot{\phi}
\end{equation}
\[
    \begin{split}
        \mathcal{L} \{f\}(s) &= \int _{0^{-}}^{\infty} f(t) e^{-st} \, \mathrm{d}t, \ 
        \mathscr{L} \{f\}(s) = \int _{0^{-}}^{\infty} f(t) e^{-st} \, \mathrm{d}t\\
        \mathcal{F} {\bigl (} f(x+x_{0}) {\bigr )} &= \mathcal{F} {\bigl (} f(x) {\bigr )} e^{2\pi i\xi x_{0}}, \ 
        \mathscr{F} {\bigl (} f(x+x_{0}) {\bigr )} = \mathscr{F} {\bigl (} f(x) {\bigr )} e^{2\pi i\xi x_{0}}
    \end{split}
\]

数学公式常用命令请见 \href{https://en.wikibooks.org/wiki/LaTeX/Mathematics}{WiKibook Mathematics}。artracom.sty中对一些常用数据类型如矢量矩阵等进行了封装,这样的好处是如有一天需要修改矢量的显示形式,只需单独修改artracom.sty中的矢量定义即可实现全文档的修改。

\subsection{数学环境}

\begin{axiom}
   这是一个公理。 
\end{axiom}
\begin{theorem}
   这是一个定理。 
\end{theorem}
\begin{lemma}
   这是一个引理。 
\end{lemma}
\begin{corollary}
   这是一个推论。 
\end{corollary}
\begin{assertion}
   这是一个断言。 
\end{assertion}
\begin{proposition}
   这是一个命题。 
\end{proposition}
\begin{proof}
    这是一个证明。
\end{proof}
\begin{definition}
    这是一个定义。
\end{definition}
\begin{example}
    这是一个例子。
\end{example}
\begin{remark}
    这是一个注。
\end{remark}


\subsection{算法}

如见算法~\ref{alg:euclid},详细使用方法请参见文档 \href{https://ctan.org/pkg/algorithmicx?lang=en}{algorithmicx}。

\begin{algorithm}[!htbp]
    \small
    \caption{Euclid's algorithm}\label{alg:euclid}
    \begin{algorithmic}[1]
        \Procedure{Euclid}{$a,b$}\Comment{The g.c.d. of a and b}
        \State $r\gets a\bmod b$
        \While{$r\not=0$}\Comment{We have the answer if r is 0}
        \State $a\gets b$
        \State $b\gets r$
        \State $r\gets a\bmod b$
        \EndWhile\label{euclidendwhile}
        \State \textbf{return} $b$\Comment{The gcd is b}
        \EndProcedure
    \end{algorithmic}
\end{algorithm}

\subsection{参考文献引用}

参考文献引用过程以实例进行介绍,假设需要引用名为"Document Preparation System"的文献,步骤如下:

1)使用Google Scholar搜索Document Preparation System,在目标条目下点击Cite,展开后选择Import into BibTeX打开此文章的BibTeX索引信息,将它们copy添加到ref.bib文件中(此文件位于Biblio文件夹下)。

2)索引第一行 \verb|@article{lamport1986document,|中 \verb|lamport1986document| 即为此文献的label (\textbf{中文文献也必须使用英文label},一般遵照:姓氏拼音+年份+标题第一字拼音的格式),想要在论文中索引此文献,有两种索引类型:

文本类型:\verb|\citet{lamport1986document}|。正如此处所示 \citet{lamport1986document}; 

括号类型:\verb|\citep{lamport1986document}|。正如此处所示 \citep{lamport1986document}。

\textbf{多文献索引用英文逗号隔开}:

\verb|\citep{lamport1986document, chu2004tushu, chen2005zhulu}|。正如此处所示 \citep{lamport1986document, chu2004tushu, chen2005zhulu}

更多例子如:

\citet{walls2013drought} 根据 \citet{betts2005aging} 的研究,首次提出...。其中关于... \citep{walls2013drought, betts2005aging},是当前中国...得到迅速发展的研究领域 \citep{chen1980zhongguo, bravo1990comparative}。引用同一著者在同一年份出版的多篇文献时,在出版年份之后用
英文小写字母区别,如:\citep{yuan2012lana, yuan2012lanb, yuan2012lanc} 和 \citet{yuan2012lana, yuan2012lanb, yuan2012lanc}。同一处引用多篇文献时,按出版年份由近及远依次标注。例如 \citep{chen1980zhongguo, stamerjohanns2009mathml, hls2012jinji, niu2013zonghe}。

使用著者-出版年制(authoryear)式参考文献样式时,中文文献必须在BibTeX索引信息的 \textbf{key} 域(请参考ref.bib文件)填写作者姓名的拼音,才能使得文献列表按照拼音排序。参考文献表中的条目(不排序号),先按语种分类排列,语种顺 序是:中文、日文、英文、俄文、其他文种。然后,中文按汉语拼音字母顺序排列,日文按第一著者的姓氏笔画排序,西文和 俄文按第一著者姓氏首字母顺序排列。如中 \citep{niu2013zonghe}、日 \citep{Bohan1928}、英 \citep{stamerjohanns2009mathml}、俄 \citep{Dubrovin1906}。

如此,即完成了文献的索引,请查看下本文档的参考文献一章,看看是不是就是这么简单呢?是的,就是这么简单!

不同文献样式和引用样式,如著者-出版年制(authoryear)、顺序编码制(numbers)、上标顺序编码制(super)可在Thesis.tex中对artratex.sty调用实现,详见 \href{https://github.com/mohuangrui/ucasthesis/wiki}{ucasthesis 知识小站之文献样式}

%若在上标顺序编码制(super)模式下,希望在特定位置将上标改为嵌入式标,可使用 \citetns{niu2013zonghe,stamerjohanns2009mathml} 和 \citepns{niu2013zonghe,stamerjohanns2009mathml}。

参考文献索引的更多知识,请见 \href{https://en.wikibooks.org/wiki/LaTeX/Bibliography_Management}{WiKibook Bibliography}。\nocite{*}% 使文献列表显示所有参考文献(包括未引用文献)

\section{常见使用问题}\label{sec:qa}

\begin{enumerate}
    \item 模板在发布前,已在Windows,Linux,MacOS系统上测试通过。下载模板后,若编译出现错误,则请见 \href{https://github.com/mohuangrui/ucasthesis/wiki}{ucasthesis知识小站} 的 \href{https://github.com/mohuangrui/ucasthesis/wiki/%E7%BC%96%E8%AF%91%E6%8C%87%E5%8D%97}{编译指南}。

    \item 模板文档的编码为UTF-8编码。所有文件都必须采用UTF-8编码,否则编译后生成的文档将出现乱码文本。若出现文本编辑器无法打开文档或打开文档乱码的问题,请检查编辑器对UTF-8编码的支持。如果使用WinEdt作为文本编辑器(\textbf{不推荐使用}),应在其Options -> Preferences -> wrapping选项卡下将两种Wrapping Modes中的内容:
        
        TeX;HTML;ANSI;ASCII|DTX...
        
        修改为:TeX;\textbf{UTF-8|ACP;}HTML;ANSI;ASCII|DTX...
        
        同时,取消Options -> Preferences -> Unicode中的Enable ANSI Format。

    \item 推荐选择xelatex或lualatex编译引擎编译中文文档。编译脚本的默认设定为xelatex编译引擎。你也可以选择不使用脚本编译,如直接使用 \LaTeX{}文本编辑器编译。注:\LaTeX{}文本编辑器编译的默认设定为pdflatex编译引擎,若选择xelatex或lualatex编译引擎,请进入下拉菜单选择。为正确生成引用链接和参考文献,需要进行\textbf{全编译}。

    \item Texmaker使用简介
        \begin{enumerate}
            \footnotesize
            \item 使用 Texmaker “打开 (Open)” Thesis.tex。
            \item 菜单 “选项 (Options)” -> “设置当前文档为主文档 (Define as Master Document)”
            \item 菜单 “自定义 (User)” -> “自定义命令 (User Commands)” -> “编辑自定义命令 (Edit User Commands)” -> 左侧选择 “command 1”,右侧 “菜单项 (Menu Item)” 填入 Auto Build -> 点击下方“向导 (Wizard)” -> “添加 (Add)”: xelatex + bibtex + xelatex + xelatex + pdf viewer -> 点击“完成 (OK)”
            \item 使用 Auto Build 编译带有未生成引用链接的源文件,可以仅使用 xelatex 编译带有已经正确生成引用链接的源文件。
            \item 编译完成,“查看(View)” PDF,在PDF中 “ctrl+click” 可链接到相对应的源文件。
        \end{enumerate}
    
    \item 模版的设计可能地考虑了适应性。致谢等所有条目都是通过最为通用的

        \verb+\chapter{item name}+  and \verb+\section*{item name}+

        来显式实现的 (请观察Backmatter.tex),从而可以随意添加,放置,和修改,如同一般章节。对于图表目录名称则可在ucasthesis.cfg中进行修改。

    \item 设置文档样式: 在artratex.sty中搜索关键字定位相应命令,然后修改
        \begin{enumerate}
            \item 正文行距:启用和设置 \verb|\linespread{1.5}|,默认1.5倍行距。
            \item 参考文献行距:修改 \verb|\setlength{\bibsep}{0.0ex}|
            \item 目录显示级数:修改 \verb|\setcounter{tocdepth}{2}|
            \item 文档超链接的颜色及其显示:修改 \verb|\hypersetup|
        \end{enumerate}


\end{enumerate}


