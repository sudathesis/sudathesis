\chapter{模板文件介绍}\label{chap:texintro}

基于文件分类管理原则便于修改与操作的属性,本模板对sudathesis的框架和文件体系进行分类系统管理。对于初次使用者,\LaTeX{}编译后的众多的文件目录不知所云,在阅读完模板文件介绍后,将会轻松使用。除此之外,\LaTeX{}的使用方法可以通过阅读相关资料如\LaTeX{} Wikibook \citep{wikibook2014latex} 增添使用技巧。


\section{安装与操作}

\begin{enumerate}
    \item 安装软件:根据所用操作系统和章节~\ref{sec:system}中的信息安装\LaTeX{}编译环境。
    \item 获取模板:下载 \href{https://github.com/sudathesis/sudathesis}{\texttt{sudathesis}} 模板并解压。sudathesis模板不仅提供了相应的类文件,同时也提供了包括参考文献等在内的完成学位论文的一切要素,所以,下载时,推荐下载整个sudathesis文件夹,而不是单独的文档类。
    \item 编译模板:
        \begin{enumerate}
            \item Windows:双击运行artratex.bat脚本。
            \item Linux或MacOS: { \verb|terminal| -> \verb|chmod +x ./artratex.sh| -> \verb|./artratex.sh xa|}
            \item \color{red}任意系统:都可使用\LaTeX{}编辑器打开Thesis.tex文件并选择xelatex编译引擎进行编译。\\
            具体编译流程为 Xelatex -> Bibtex -> Xelatex -> Xelatex,即后文所提方法。
        \end{enumerate}
    \item 错误处理:若编译中遇到了问题,请先查看“常见问题”(章节~\ref{sec:qa})。
\end{enumerate}

编译完成即可获得本PDF说明文档。而这也完成了学习使用sudathesis撰写论文的一半进程。

\section{文档目录简介}

\subsection{Thesis.tex}

Thesis.tex为主文档,其设计和规划了论文的整体框架,通过对其的阅读可以了解整个论文框架的搭建。

\subsection{编译脚本}

\begin{itemize}
    \item Windows:双击Dos脚本artratex.bat可得全编译后的PDF文档,其存在是为了帮助不了解\LaTeX{}编译过程的初学者跨过编译这第一道坎,请勿通过邮件传播和接收此脚本,以防范Dos脚本的潜在风险。
    \item Linux或MacOS:在terminal中运行
        \begin{itemize}
            \item \verb|./artratex.sh xa|:获得全编译后的PDF文档
            \item \verb|./artratex.sh x|:快速编译,不会生成文献引用
        \end{itemize}
\end{itemize}

全编译指运行 \verb|Xelatex+Bibtex+Xelatex+Xelatex| 以正确生成所有的引用链接,如目录,参考文献及引用等。在写作过程中若无添加新的引用,则可用快速编译,即只运行一遍\LaTeX{}编译引擎以减少编译时间。

\subsection{Tmp文件夹}

运行编译脚本后,编译所生成的文档皆存于Tmp文件夹内,包括编译得到的PDF文档,其存在是为了保持工作空间的整洁,因为好的心情是很重要的。

\subsection{Style文件夹}

包含ucasthesis文档类的定义文件和配置文件,通过对它们的修改可以实现特定的模版设定。

\begin{enumerate}
    \item sudathesis.cls:文档类定义文件,论文的最核心的格式即通过它来定义的。
    \item sudathesis.cfg:文档类配置文件,设定如目录显示为“目~录”而非“目录”。
    \item artratex.sty: 常用宏包及文档设定,如参考文献样式、文献引用样式、页眉页脚设定等。这些功能具有开关选项,常只需在Thesis.tex中进行启用即可,一般无需修改artratex.sty本身。
    \item artracom.sty:自定义命令以及添加宏包的推荐放置位置。
\end{enumerate}

\subsection{Tex文件夹}

文件夹内为论文的所有实体内容,正常情况下,这也是\textbf{使用sudathesis撰写学位论文时,主要关注和修改的一个位置,注:所有文件都必须采用UTF-8编码,否则编译后将出现乱码文本},详细分类介绍如下:

\begin{itemize}
    \item Frontinfo.tex:为论文中英文封面信息。\textbf{论文封面会根据英文学位名称如Bachelor,Master,Doctor, Postdoctor 自动切换为相应的格式}。
    \item Frontmatter.tex:为论文前言内容如中英文摘要等。
    \item Mainmatter.tex:索引需要出现的Chapter。开始写论文时,可以只索引当前章节,以快速编译查看,当论文完成后,再对所有章节进行索引即可。
    \item Chap{\_}xxx.tex:为论文主体的各章,可根据需要添加和撰写。\textbf{添加新章时,可拷贝一个已有的章文件再重命名,以继承文档的 UTF8 编码}。
    \item Appendix.tex:为附录内容。
    \item Backmatter.tex:为发表文章信息和致谢部分等。
\end{itemize}

\subsection{Img文件夹}

用于放置论文中所需要的图类文件,支持格式有:.eps, .jpg, .png, .pdf。其中,\verb|logo.png|苏州大学校徽、\verb|name.png|苏州大学校名图片格式。若图片众多,也可为各章节图片建立子目录;但若命名规则合理,图片查询亦是十分方便。

\subsection{Biblio文件夹}

\begin{enumerate}
    \item ref.bib:参考文献信息库。
\end{enumerate}


